% Options for packages loaded elsewhere
\PassOptionsToPackage{unicode}{hyperref}
\PassOptionsToPackage{hyphens}{url}
%
\documentclass[
  11pt,
]{book}
\title{A Minimal Book Example with ARCA Template}
\usepackage{etoolbox}
\makeatletter
\providecommand{\subtitle}[1]{% add subtitle to \maketitle
  \apptocmd{\@title}{\par {\large #1 \par}}{}{}
}
\makeatother
\subtitle{A Handbook for \ldots{}}
\author{ARCA}
\date{2022-04-21}

\usepackage{amsmath,amssymb}
\usepackage{lmodern}
\usepackage{iftex}
\ifPDFTeX
  \usepackage[T1]{fontenc}
  \usepackage[utf8]{inputenc}
  \usepackage{textcomp} % provide euro and other symbols
\else % if luatex or xetex
  \usepackage{unicode-math}
  \defaultfontfeatures{Scale=MatchLowercase}
  \defaultfontfeatures[\rmfamily]{Ligatures=TeX,Scale=1}
\fi
% Use upquote if available, for straight quotes in verbatim environments
\IfFileExists{upquote.sty}{\usepackage{upquote}}{}
\IfFileExists{microtype.sty}{% use microtype if available
  \usepackage[]{microtype}
  \UseMicrotypeSet[protrusion]{basicmath} % disable protrusion for tt fonts
}{}
\makeatletter
\@ifundefined{KOMAClassName}{% if non-KOMA class
  \IfFileExists{parskip.sty}{%
    \usepackage{parskip}
  }{% else
    \setlength{\parindent}{0pt}
    \setlength{\parskip}{6pt plus 2pt minus 1pt}}
}{% if KOMA class
  \KOMAoptions{parskip=half}}
\makeatother
\usepackage{xcolor}
\IfFileExists{xurl.sty}{\usepackage{xurl}}{} % add URL line breaks if available
\IfFileExists{bookmark.sty}{\usepackage{bookmark}}{\usepackage{hyperref}}
\hypersetup{
  pdftitle={A Minimal Book Example with ARCA Template},
  pdfauthor={ARCA},
  hidelinks,
  pdfcreator={LaTeX via pandoc}}
\urlstyle{same} % disable monospaced font for URLs
\usepackage{color}
\usepackage{fancyvrb}
\newcommand{\VerbBar}{|}
\newcommand{\VERB}{\Verb[commandchars=\\\{\}]}
\DefineVerbatimEnvironment{Highlighting}{Verbatim}{commandchars=\\\{\}}
% Add ',fontsize=\small' for more characters per line
\usepackage{framed}
\definecolor{shadecolor}{RGB}{248,248,248}
\newenvironment{Shaded}{\begin{snugshade}}{\end{snugshade}}
\newcommand{\AlertTok}[1]{\textcolor[rgb]{0.94,0.16,0.16}{#1}}
\newcommand{\AnnotationTok}[1]{\textcolor[rgb]{0.56,0.35,0.01}{\textbf{\textit{#1}}}}
\newcommand{\AttributeTok}[1]{\textcolor[rgb]{0.77,0.63,0.00}{#1}}
\newcommand{\BaseNTok}[1]{\textcolor[rgb]{0.00,0.00,0.81}{#1}}
\newcommand{\BuiltInTok}[1]{#1}
\newcommand{\CharTok}[1]{\textcolor[rgb]{0.31,0.60,0.02}{#1}}
\newcommand{\CommentTok}[1]{\textcolor[rgb]{0.56,0.35,0.01}{\textit{#1}}}
\newcommand{\CommentVarTok}[1]{\textcolor[rgb]{0.56,0.35,0.01}{\textbf{\textit{#1}}}}
\newcommand{\ConstantTok}[1]{\textcolor[rgb]{0.00,0.00,0.00}{#1}}
\newcommand{\ControlFlowTok}[1]{\textcolor[rgb]{0.13,0.29,0.53}{\textbf{#1}}}
\newcommand{\DataTypeTok}[1]{\textcolor[rgb]{0.13,0.29,0.53}{#1}}
\newcommand{\DecValTok}[1]{\textcolor[rgb]{0.00,0.00,0.81}{#1}}
\newcommand{\DocumentationTok}[1]{\textcolor[rgb]{0.56,0.35,0.01}{\textbf{\textit{#1}}}}
\newcommand{\ErrorTok}[1]{\textcolor[rgb]{0.64,0.00,0.00}{\textbf{#1}}}
\newcommand{\ExtensionTok}[1]{#1}
\newcommand{\FloatTok}[1]{\textcolor[rgb]{0.00,0.00,0.81}{#1}}
\newcommand{\FunctionTok}[1]{\textcolor[rgb]{0.00,0.00,0.00}{#1}}
\newcommand{\ImportTok}[1]{#1}
\newcommand{\InformationTok}[1]{\textcolor[rgb]{0.56,0.35,0.01}{\textbf{\textit{#1}}}}
\newcommand{\KeywordTok}[1]{\textcolor[rgb]{0.13,0.29,0.53}{\textbf{#1}}}
\newcommand{\NormalTok}[1]{#1}
\newcommand{\OperatorTok}[1]{\textcolor[rgb]{0.81,0.36,0.00}{\textbf{#1}}}
\newcommand{\OtherTok}[1]{\textcolor[rgb]{0.56,0.35,0.01}{#1}}
\newcommand{\PreprocessorTok}[1]{\textcolor[rgb]{0.56,0.35,0.01}{\textit{#1}}}
\newcommand{\RegionMarkerTok}[1]{#1}
\newcommand{\SpecialCharTok}[1]{\textcolor[rgb]{0.00,0.00,0.00}{#1}}
\newcommand{\SpecialStringTok}[1]{\textcolor[rgb]{0.31,0.60,0.02}{#1}}
\newcommand{\StringTok}[1]{\textcolor[rgb]{0.31,0.60,0.02}{#1}}
\newcommand{\VariableTok}[1]{\textcolor[rgb]{0.00,0.00,0.00}{#1}}
\newcommand{\VerbatimStringTok}[1]{\textcolor[rgb]{0.31,0.60,0.02}{#1}}
\newcommand{\WarningTok}[1]{\textcolor[rgb]{0.56,0.35,0.01}{\textbf{\textit{#1}}}}
\usepackage{longtable,booktabs,array}
\usepackage{calc} % for calculating minipage widths
% Correct order of tables after \paragraph or \subparagraph
\usepackage{etoolbox}
\makeatletter
\patchcmd\longtable{\par}{\if@noskipsec\mbox{}\fi\par}{}{}
\makeatother
% Allow footnotes in longtable head/foot
\IfFileExists{footnotehyper.sty}{\usepackage{footnotehyper}}{\usepackage{footnote}}
\makesavenoteenv{longtable}
\usepackage{graphicx}
\makeatletter
\def\maxwidth{\ifdim\Gin@nat@width>\linewidth\linewidth\else\Gin@nat@width\fi}
\def\maxheight{\ifdim\Gin@nat@height>\textheight\textheight\else\Gin@nat@height\fi}
\makeatother
% Scale images if necessary, so that they will not overflow the page
% margins by default, and it is still possible to overwrite the defaults
% using explicit options in \includegraphics[width, height, ...]{}
\setkeys{Gin}{width=\maxwidth,height=\maxheight,keepaspectratio}
% Set default figure placement to htbp
\makeatletter
\def\fps@figure{htbp}
\makeatother
\setlength{\emergencystretch}{3em} % prevent overfull lines
\providecommand{\tightlist}{%
  \setlength{\itemsep}{0pt}\setlength{\parskip}{0pt}}
\setcounter{secnumdepth}{5}
\newlength{\cslhangindent}
\setlength{\cslhangindent}{1.5em}
\newlength{\csllabelwidth}
\setlength{\csllabelwidth}{3em}
\newlength{\cslentryspacingunit} % times entry-spacing
\setlength{\cslentryspacingunit}{\parskip}
\newenvironment{CSLReferences}[2] % #1 hanging-ident, #2 entry spacing
 {% don't indent paragraphs
  \setlength{\parindent}{0pt}
  % turn on hanging indent if param 1 is 1
  \ifodd #1
  \let\oldpar\par
  \def\par{\hangindent=\cslhangindent\oldpar}
  \fi
  % set entry spacing
  \setlength{\parskip}{#2\cslentryspacingunit}
 }%
 {}
\usepackage{calc}
\newcommand{\CSLBlock}[1]{#1\hfill\break}
\newcommand{\CSLLeftMargin}[1]{\parbox[t]{\csllabelwidth}{#1}}
\newcommand{\CSLRightInline}[1]{\parbox[t]{\linewidth - \csllabelwidth}{#1}\break}
\newcommand{\CSLIndent}[1]{\hspace{\cslhangindent}#1}
%----------------------------%
%----    preamble.tex    ----%
%----------------------------%

\usepackage{booktabs}
\usepackage{amsthm}
\makeatletter
\def\thm@space@setup{%
  \thm@preskip=8pt plus 2pt minus 4pt
  \thm@postskip=\thm@preskip
}
\makeatother

%----    layout    ----%

% change margins
\usepackage[outer = 105pt, inner = 85pt]{geometry}
\usepackage{afterpage}
\voffset 5pt
\headsep 29pt
\footskip 35pt
\textheight 590pt
\usepackage{layout}% http://ctan.org/pkg/layouts

% caption
\usepackage[labelfont=bf]{caption}


%---- page numbers header and footer
\usepackage{xcolor}

\definecolor{ctcolorgraylight}{RGB}{153, 153, 153}
\definecolor{ctcolorgray}{RGB}{120, 120, 120}

\usepackage{fancyhdr}
\usepackage[fit]{truncate}

% Redefine the plain page style
\fancypagestyle{plain}{%
  \fancyhf{}%
  \fancyfoot[LE,RO]{\textcolor{ctcolorgray}{\thepage}}
  \renewcommand{\headrulewidth}{0pt}% Line at the header invisible
}

% fancy style
\fancypagestyle{myfancy}{%
  \fancyhf{}
  \fancyhead[LE]{\footnotesize\textit{\nouppercase{\truncate{0.98\headwidth}{\textcolor{ctcolorgray}{\rightmark}}}}}
  \fancyhead[RO]{\footnotesize\textit{\truncate{0.98\headwidth}{\textcolor{ctcolorgray}{\leftmark}}}}
  \fancyfoot[LE,RO]{\textcolor{ctcolorgray}{\thepage}}
  \renewcommand{\headrulewidth}{1.5pt}
  \renewcommand{\headrule}{\hbox to\headwidth{%
      \color{ctcolorgraylight}\leaders\hrule height \headrulewidth\hfill}}
}

% part page style
\newcommand{\cpart}[1]{
\cleardoublepage
\stepcounter{part}
\pagestyle{empty}
\begin{flushright}
  \vspace*{2.5cm}
  {\fontfamily{pnc}\fontsize{75}{80}\selectfont%
   \textcolor{ctcolorgraylight}{\textbf{Part \thepart}}\par}%
  \vspace*{1cm}
  {\Huge\textbf{#1}}%
  \end{flushright}
\addcontentsline{toc}{part}{Part \thepart\hspace{1em}#1}
\cleardoublepage
\pagestyle{plain}
}

\usepackage{titletoc}
\titlecontents{part}[0pt]{\bfseries\protect\addvspace{20pt}\titlerule\addvspace{.7ex}}%
{}{}%% numbered/numberless formatting
{}%% to be replaced with {} if you don't want any page number for parts
[\addvspace{0.7ex}\titlerule\addvspace{1.5ex}]

%---- chapter title
\usepackage[newcentury]{quotchap}

\makeatletter
\renewcommand*{\chapnumfont}{%
  \usefont{T1}{\@defaultcnfont}{b}{n}\fontsize{120}{150}\selectfont% Default: 100/130
  \color{chaptergrey}%
}
\makeatother

%----    define infoboxes    ----%
\usepackage[skins, most]{tcolorbox}

% colors
\definecolor{background}{HTML}{fcfcfc}
\definecolor{tip-text}{HTML}{e7b002}
\definecolor{tip-line}{HTML}{fdce38}
\definecolor{warning-text}{HTML}{b06336}
\definecolor{warning-line}{HTML}{c97d50}
\definecolor{deffun-text}{HTML}{0b797e}
\definecolor{deffun-line}{HTML}{6CC2C9}
\definecolor{design-text}{HTML}{7c972e}
\definecolor{design-line}{HTML}{a7c84a}
\definecolor{trick-text}{HTML}{8c3031}
\definecolor{trick-line}{HTML}{A3595A}

\newtcolorbox{mybox}[1][black]{
  colback=background,
  coltext=black,
  colframe=#1,
  boxsep=5pt,
  arc=4pt,
  breakable}

% tip
\newenvironment{tip}[1][Title]
  {
  \setlength{\fboxsep}{1em}
  \begin{mybox}[tip-line]
    \raisebox{-.2\height}{\includegraphics[height=.6cm]{assets/images/lightbulb.png}} \large \textcolor{tip-text}{Tip-Box: #1\\\vspace*{.5em}}\normalsize
    }
    {
  \end{mybox}
  }

% warnings
\newenvironment{warning}[1][Title]
  {
  \setlength{\fboxsep}{1em}
  \begin{mybox}[warning-line]
    \raisebox{-.2\height}{\includegraphics[height=.6cm]{assets/images/gotcha.png}} \large \textcolor{warning-text}{Warning-Box: #1\\\vspace*{.5em}}\normalsize
    }
    {
  \end{mybox}
  }

% deffun
\newenvironment{deffun}[1][Title]
  {
  \setlength{\fboxsep}{1em}
  \begin{mybox}[deffun-line]
    \raisebox{-.2\height}{\includegraphics[height=.6cm]{assets/images/gears.png}} \large \textcolor{deffun-text}{Definition-Box: #1\\\vspace*{.5em}}\normalsize
    }
    {
  \end{mybox}
  }

% design
\newenvironment{design}[1][Title]
  {
  \setlength{\fboxsep}{1em}
  \begin{mybox}[design-line]
    \raisebox{-.2\height}{\includegraphics[height=.6cm]{assets/images/design.png}} \large \textcolor{design-text}{Design-Box: #1\\\vspace*{.5em}}\normalsize
    }
    {
  \end{mybox}
  }

% trick
\newenvironment{trick}[1][Title]
  {
  \setlength{\fboxsep}{1em}
  \begin{mybox}[trick-line]
    \raisebox{-.2\height}{\includegraphics[height=.6cm]{assets/images/hat.png}} \large \textcolor{trick-text}{Trick-Box: #1\\\vspace*{.5em}}\normalsize
    }
    {
  \end{mybox}
  }

%----    quote


\definecolor{block-gray}{gray}{0.95}


\newtcolorbox{line-left}{%
    colback=white,
    % grow to right by=-10mm,
    % grow to left by=-10mm,
    boxrule=0pt,
    boxsep=0pt,
    breakable,
    enhanced jigsaw,
    borderline west={2pt}{0pt}{gray},
    borderline north={0pt}{0pt}{white},
    borderline south={0pt}{0pt}{white},
    % title={#2\par},
    % colbacktitle={block-gray},
    % coltitle={black},
    % fonttitle={\large\bfseries},
    % attach title to upper={},
    % #1,
}

% \renewenvironment{quote}
%                {\begin{line-left}}
%                {\end{line-left}}

\renewenvironment{quote}
               {\list{}{\rightmargin\leftmargin}%
                \item\relax\begin{line-left}\setlength{\parskip}{1em}}
               {\end{line-left}\endlist}

%----

\usepackage{hyperref}

\usepackage{titlepic}
\titlepic{\includegraphics[width=.8\textwidth]{images/arca-logo.pdf}}
\ifLuaTeX
  \usepackage{selnolig}  % disable illegal ligatures
\fi

\begin{document}
\maketitle

\pagestyle{empty}
\clearpage
\frontmatter
\pagestyle{plain}

{
\setcounter{tocdepth}{1}
\tableofcontents
}
\mainmatter
\pagestyle{myfancy}

\hypertarget{prerequisites}{%
\chapter*{Prerequisites}\label{prerequisites}}
\addcontentsline{toc}{chapter}{Prerequisites}

This is a \emph{sample} book written in \textbf{Markdown}. You can use anything that Pandoc's Markdown supports, e.g., a math equation \(a^2 + b^2 = c^2\).

The \textbf{bookdown} package can be installed from CRAN or Github:

\begin{Shaded}
\begin{Highlighting}[]
\FunctionTok{install.packages}\NormalTok{(}\StringTok{"bookdown"}\NormalTok{)}
\CommentTok{\# or the development version}
\CommentTok{\# devtools::install\_github("rstudio/bookdown")}
\end{Highlighting}
\end{Shaded}

Remember each Rmd file contains one and only one chapter, and a chapter is defined by the first-level heading \texttt{\#}.

To compile this example to PDF, you need XeLaTeX. You are recommended to install TinyTeX (which includes XeLaTeX): \url{https://yihui.name/tinytex/}.

\hypertarget{arca-template}{%
\section*{ARCA Template}\label{arca-template}}
\addcontentsline{toc}{section}{ARCA Template}

ARCA Template is based on \href{https://github.com/rstudio/bookdown-demo}{Rstudio Bookdown-demo} released under \href{https://creativecommons.org/licenses/by-sa/4.0/}{CC BY-SA 4.0 License}

Icons by \href{https://rstudio4edu.github.io/rstudio4edu-book/}{rstudio4edu} released under \href{https://creativecommons.org/licenses/by-nc/2.0/}{CC BY-NC 2.0 License}.

\hypertarget{deploy-github-actions}{%
\section*{Deploy Github Actions}\label{deploy-github-actions}}
\addcontentsline{toc}{section}{Deploy Github Actions}

Follow tutorial at \url{https://medium.com/@delucmat/how-to-publish-bookdown-projects-with-github-actions-on-github-pages-6e6aecc7331e} but note that github action \url{https://github.com/Cecilapp/GitHub-Pages-deploy} is slightly changed so we adapted de code. In particular now we have as last action:

\begin{verbatim}
- name: Deploy to GitHub Pages
       uses: Cecilapp/GitHub-Pages-deploy@v3
       env:
         GITHUB_TOKEN: ${{ secrets.GITHUB_TOKEN }}
       with:
         email: ${{ secrets.EMAIL }}
         build_dir: _site/
\end{verbatim}

Moreover, we also installed \texttt{tinytex} and specified \texttt{rmarkdown::render\_site(encoding\ =\ "UTF-8")} in the first job to obtain pdf and epub available versions as well.

\hypertarget{html-and-latex}{%
\section*{HTML and LaTeX}\label{html-and-latex}}
\addcontentsline{toc}{section}{HTML and LaTeX}

Remember that as the output is compiled to create a website and a PDF you have to take care of defining options and environments in both cases. See official documeentation \url{https://bookdown.org/yihui/bookdown/}

\hypertarget{intro}{%
\chapter{Introduction}\label{intro}}

You can label chapter and section titles using \texttt{\{\#label\}} after them, e.g., we can reference Chapter \ref{intro}. If you do not manually label them, there will be automatic labels anyway, e.g., Chapter \ref{methods}.

Figures and tables with captions will be placed in \texttt{figure} and \texttt{table} environments, respectively.

\begin{Shaded}
\begin{Highlighting}[]
\FunctionTok{par}\NormalTok{(}\AttributeTok{mar =} \FunctionTok{c}\NormalTok{(}\DecValTok{4}\NormalTok{, }\DecValTok{4}\NormalTok{, .}\DecValTok{1}\NormalTok{, .}\DecValTok{1}\NormalTok{))}
\FunctionTok{plot}\NormalTok{(pressure, }\AttributeTok{type =} \StringTok{\textquotesingle{}b\textquotesingle{}}\NormalTok{, }\AttributeTok{pch =} \DecValTok{19}\NormalTok{)}
\end{Highlighting}
\end{Shaded}

\begin{figure}[h]

{\centering \includegraphics[width=0.8\linewidth]{bookdown-arca-demo_files/figure-latex/nice-fig-1} 

}

\caption{Here is a nice figure!}\label{fig:nice-fig}
\end{figure}

Reference a figure by its code chunk label with the \texttt{fig:} prefix, e.g., see Figure \ref{fig:nice-fig}. Similarly, you can reference tables generated from \texttt{knitr::kable()}, e.g., see Table \ref{tab:nice-tab}.

\begin{Shaded}
\begin{Highlighting}[]
\FunctionTok{kable}\NormalTok{(}
  \FunctionTok{head}\NormalTok{(iris, }\DecValTok{5}\NormalTok{), }\AttributeTok{caption =} \StringTok{\textquotesingle{}Here is a nice table!\textquotesingle{}}\NormalTok{,}
  \AttributeTok{booktabs =} \ConstantTok{TRUE}
\NormalTok{) }\SpecialCharTok{\%\textgreater{}\%} 
  \FunctionTok{kable\_styling}\NormalTok{(}\AttributeTok{bootstrap\_options =} \FunctionTok{c}\NormalTok{(}\StringTok{"striped"}\NormalTok{, }\StringTok{"hover"}\NormalTok{), }
                \AttributeTok{full\_width =} \ConstantTok{FALSE}\NormalTok{)}
\end{Highlighting}
\end{Shaded}

\begin{table}

\caption{\label{tab:nice-tab}Here is a nice table!}
\centering
\begin{tabular}[t]{rrrrl}
\toprule
Sepal.Length & Sepal.Width & Petal.Length & Petal.Width & Species\\
\midrule
5.1 & 3.5 & 1.4 & 0.2 & setosa\\
4.9 & 3.0 & 1.4 & 0.2 & setosa\\
4.7 & 3.2 & 1.3 & 0.2 & setosa\\
4.6 & 3.1 & 1.5 & 0.2 & setosa\\
5.0 & 3.6 & 1.4 & 0.2 & setosa\\
\bottomrule
\end{tabular}
\end{table}

You can write citations, too. For example, we are using the \textbf{bookdown} package (\protect\hyperlink{ref-R-bookdown}{Xie, 2022}) in this sample book, which was built on top of R Markdown and \textbf{knitr} (\protect\hyperlink{ref-xie2015}{Xie, 2015}).

\hypertarget{r-markdown}{%
\section{R Markdown}\label{r-markdown}}

This is an R Markdown document. Markdown is a simple formatting syntax for authoring HTML, PDF, and MS Word documents. For more details on using R Markdown see \url{http://rmarkdown.rstudio.com}.

When you click the \textbf{Knit} button a document will be generated that includes both content as well as the output of any embedded R code chunks within the document. You can embed an R code chunk like this:

\begin{Shaded}
\begin{Highlighting}[]
\FunctionTok{summary}\NormalTok{(cars)}
\end{Highlighting}
\end{Shaded}

\begin{verbatim}
##      speed           dist       
##  Min.   : 4.0   Min.   :  2.00  
##  1st Qu.:12.0   1st Qu.: 26.00  
##  Median :15.0   Median : 36.00  
##  Mean   :15.4   Mean   : 42.98  
##  3rd Qu.:19.0   3rd Qu.: 56.00  
##  Max.   :25.0   Max.   :120.00
\end{verbatim}

\hypertarget{including-plots}{%
\subsection{Including Plots}\label{including-plots}}

You can also embed plots, for example:

\begin{center}\includegraphics{bookdown-arca-demo_files/figure-latex/pressure-1} \end{center}

Note that the \texttt{echo\ =\ FALSE} parameter was added to the code chunk to prevent printing of the R code that generated the plot.

\hypertarget{content-hyperlinks}{%
\section{Content Hyperlinks}\label{content-hyperlinks}}

\hypertarget{my-section}{%
\subsection{Sections}\label{my-section}}

See Section \ref{my-section}

\hypertarget{figures}{%
\subsection{Figures}\label{figures}}

\hypertarget{pictures}{%
\subsubsection{Pictures}\label{pictures}}

See Figure \ref{fig:arca-logo}. Note: in chunks name do not use ``\_'' but use ``-'' instead. \texttt{\textbackslash{}@ref(fig:arca\_logo)} do not work, \texttt{\textbackslash{}@ref(fig:arca-logo)} works properly.

\begin{figure}[h]

{\centering \includegraphics[width=0.33\linewidth]{images/arca-logo} 

}

\caption{Logo ARCA}\label{fig:arca-logo}
\end{figure}

\hypertarget{plots}{%
\subsubsection{Plots}\label{plots}}

See Figure \ref{fig:my-plot}

\begin{figure}[h]

{\centering \includegraphics[width=0.5\linewidth]{bookdown-arca-demo_files/figure-latex/my-plot-1} 

}

\caption{Random numbers}\label{fig:my-plot}
\end{figure}

\hypertarget{tables}{%
\subsection{Tables}\label{tables}}

See r-package \texttt{kableExtra} documentation (\href{https://cran.r-project.org/web/packages/kableExtra/vignettes/awesome_table_in_html.html}{link}).

See Tabele \ref{tab:cars-table}

\begin{table}

\caption{\label{tab:cars-table}Here is a nice table!}
\centering
\begin{tabular}[t]{rrrrl}
\toprule
Sepal.Length & Sepal.Width & Petal.Length & Petal.Width & Species\\
\midrule
5.1 & 3.5 & 1.4 & 0.2 & setosa\\
4.9 & 3.0 & 1.4 & 0.2 & setosa\\
4.7 & 3.2 & 1.3 & 0.2 & setosa\\
4.6 & 3.1 & 1.5 & 0.2 & setosa\\
5.0 & 3.6 & 1.4 & 0.2 & setosa\\
\bottomrule
\end{tabular}
\end{table}

\hypertarget{apa-cls}{%
\section{APA cls}\label{apa-cls}}

We are using apa 7 cls format. Citation Syntax (\href{https://rmarkdown.rstudio.com/authoring_bibliographies_and_citations.html}{link}).

\hypertarget{infobox}{%
\section{Infobox}\label{infobox}}

Icons by \href{https://rstudio4edu.github.io/rstudio4edu-book/}{rstudio4edu} released under \href{https://creativecommons.org/licenses/by-nc/2.0/}{CC BY-NC 2.0 License}. Remember to include an \emph{Attributions} section in the book and repository's README file.

\begin{tip}[My title]
Lorem ipsum dolor sit amet consectetur adipisicing elit. Maxime mollitia,
molestiae quas vel sint commodi repudiandae consequuntur voluptatum laborum
numquam blanditiis harum quisquam eius sed odit fugiat iusto fuga praesentium
optio, eaque rerum!

Lorem ipsum dolor sit amet consectetur adipisicing elit. Maxime mollitia,
molestiae quas vel sint commodi repudiandae consequuntur voluptatum laborum
numquam blanditiis harum quisquam eius sed odit fugiat iusto fuga praesentium
optio, eaque rerum!

Lorem ipsum dolor sit amet consectetur adipisicing elit. Maxime mollitia,
molestiae quas vel sint commodi repudiandae consequuntur voluptatum laborum
numquam blanditiis harum quisquam eius sed odit fugiat iusto fuga praesentium
optio, eaque rerum!

Lorem ipsum dolor sit amet consectetur adipisicing elit. Maxime mollitia,
molestiae quas vel sint commodi repudiandae consequuntur voluptatum laborum
numquam blanditiis harum quisquam eius sed odit fugiat iusto fuga praesentium
optio, eaque rerum!

\end{tip}

\begin{warning}[My title]
Lorem ipsum dolor sit amet consectetur adipisicing elit. Maxime mollitia,
molestiae quas vel sint commodi repudiandae consequuntur voluptatum laborum
numquam blanditiis harum quisquam eius sed odit fugiat iusto fuga praesentium
optio, eaque rerum!

Lorem ipsum dolor sit amet consectetur adipisicing elit. Maxime mollitia,
molestiae quas vel sint commodi repudiandae consequuntur voluptatum laborum
numquam blanditiis harum quisquam eius sed odit fugiat iusto fuga praesentium
optio, eaque rerum!

Lorem ipsum dolor sit amet consectetur adipisicing elit. Maxime mollitia,
molestiae quas vel sint commodi repudiandae consequuntur voluptatum laborum
numquam blanditiis harum quisquam eius sed odit fugiat iusto fuga praesentium
optio, eaque rerum!

\end{warning}

\begin{deffun}[My title]
Lorem ipsum dolor sit amet consectetur adipisicing elit. Maxime mollitia,
molestiae quas vel sint commodi repudiandae consequuntur voluptatum laborum
numquam blanditiis harum quisquam eius sed odit fugiat iusto fuga praesentium
optio, eaque rerum!

\end{deffun}

\begin{design}[My title]
Lorem ipsum dolor sit amet consectetur adipisicing elit. Maxime mollitia,
molestiae quas vel sint commodi repudiandae consequuntur voluptatum laborum
numquam blanditiis harum quisquam eius sed odit fugiat iusto fuga praesentium
optio, eaque rerum!

\end{design}

\begin{trick}[My title]
Lorem ipsum dolor sit amet consectetur adipisicing elit. Maxime mollitia,
molestiae quas vel sint commodi repudiandae consequuntur voluptatum laborum
numquam blanditiis harum quisquam eius sed odit fugiat iusto fuga praesentium
optio, eaque rerum!

\end{trick}

\hypertarget{literature}{%
\chapter{Literature}\label{literature}}

Here is a review of existing methods.

\hypertarget{methods}{%
\chapter{Methods}\label{methods}}

We describe our methods in this chapter.

\hypertarget{applications}{%
\chapter{Applications}\label{applications}}

Some \emph{significant} applications are demonstrated in this chapter.

\hypertarget{example-one}{%
\section{Example one}\label{example-one}}

\hypertarget{example-two}{%
\section{Example two}\label{example-two}}

\cpart{Conclusions}

\pagestyle{myfancy}

\hypertarget{final-words}{%
\chapter{Final Words}\label{final-words}}

We have finished a nice book.

\hypertarget{references}{%
\chapter*{References}\label{references}}
\addcontentsline{toc}{chapter}{References}

\hypertarget{refs}{}
\begin{CSLReferences}{1}{0}
\leavevmode\vadjust pre{\hypertarget{ref-xie2015}{}}%
Xie, Y. (2015). \emph{Dynamic documents with {R} and knitr} (2nd ed.). Chapman; Hall/CRC. \url{http://yihui.name/knitr/}

\leavevmode\vadjust pre{\hypertarget{ref-R-bookdown}{}}%
Xie, Y. (2022). \emph{Bookdown: Authoring books and technical documents with r markdown}. \url{https://CRAN.R-project.org/package=bookdown}

\end{CSLReferences}

\end{document}
